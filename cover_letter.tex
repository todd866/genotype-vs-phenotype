\documentclass[11pt]{letter}
\usepackage[margin=1in]{geometry}
\usepackage{hyperref}
\usepackage{parskip}

\signature{Ian Todd\\Sydney Medical School\\University of Sydney\\itod2305@uni.sydney.edu.au}
\address{Ian Todd\\Sydney Medical School\\University of Sydney\\Sydney, NSW, Australia}

\begin{document}

\begin{letter}{Professor Abir Igamberdiev\\Editor-in-Chief\\BioSystems}

\opening{Dear Professor Igamberdiev,}

I am pleased to submit ``Genotype $\neq$ Phenotype: High-Dimensional Development, Plasticity, and the Limits of Allele Stories'' for consideration in \emph{BioSystems}.

This manuscript provides a mathematical formalization of a debate currently active in this journal. Lissek (2024, \emph{BioSystems} 247) recently proposed that malignancy can develop through ``cancer memory''---epigenetic mechanisms unleashed by initial oncogenic changes---rather than relying solely on continued mutation accumulation. My work supplies the information-theoretic dual to this biological argument: I show that when phenotype emerges from low-dimensional genetic parameters passing through a high-dimensional developmental system, allele-based and plasticity-based models become \textbf{formally non-identifiable} from aggregate observational data.

The paper introduces the ``Dimensional Gap'' ($\Delta_D$), which quantifies when this non-identifiability arises, and demonstrates it through a minimal developmental network model. The ``Twin Worlds'' experiment shows that identical genotype distributions in different environments produce patterns a naive allele model would misinterpret as genetic differences---with $F_{ST} \approx 0$ yet $P_{ST} \gg 0$. This directly explains why ``missing heritability'' in GWAS studies may reflect projection artifacts rather than missing variants.

The work responds to Sierra et al.\ (2025, \emph{Science Advances}), who documented lower cancer prevalence in cooperative mammalian species and modeled this as allele-based selection. I show their finding is equally consistent with plastic developmental policies responding to cooperative environmental cues---a distinction with significant implications for intervention strategies.

This submission continues a research program on dimensional constraints in biology previously published in \emph{BioSystems}:
\begin{itemize}
    \item Todd (2025a): ``The limits of falsifiability'' (DOI: 10.1016/j.biosystems.2025.105608)
    \item Todd (2025b): ``Timing inaccessibility and the projection bound'' (DOI: 10.1016/j.biosystems.2025.105632)
\end{itemize}

In the tradition of Rosen's critique of algorithmic biology, I demonstrate that the gene-as-algorithm assumption is not merely an approximation but a projection that actively discards causal structure.

The manuscript engages directly with recent \emph{BioSystems} publications including Lee et al.\ (2022) on phenotypic plasticity, Letsou (2024) on temporal structure in development, Corning (2022) on systems evolution, and Fontana (2023) on the development-ageing-cancer nexus. I believe it will be of strong interest to the journal's readership.

All simulation code is publicly available at \url{https://github.com/todd866/genotype-vs-phenotype}, with full documentation of the AI-assisted workflow used in manuscript preparation.

This manuscript is original and not under consideration for publication elsewhere.

\closing{Thank you for your consideration.}

\end{letter}
\end{document}
